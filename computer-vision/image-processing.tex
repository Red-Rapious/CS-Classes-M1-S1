\section{Image processing using filters and convolutions}
An image can be interpreted either as a continuous function $f(x, y)$ or as a discrete array $F_{u,v}$.

\subsection{Filters and convolution}
\subsubsection{Basic filters}
An image can be blurred using a filter, by replacing a point by the average of its neighbors. Blurring an image gives a smoother image, making it easier to compute derivatives. 

\subsubsection{Convolutions}
Given two integrable functions $f, g:\R\to\R$, we can define their convolution as:
\begin{align*}
    f*g: \R&\longrightarrow \R\\
    x&\longmapsto\int_{-\infty}^{+\infty}f(x-t)g(t)dt
\end{align*}
Note that $f*g=g*f$ using a change of variable.

This is the definition of the convolution from a continuous perspective. When dealing with images, we want to apply the convolution to a discrete array.
\begin{equation*}
    R_{i,j} = (F*G)_{i,j} = \sum_{u,v} F_{i-u, j-v}G_{u,v}
\end{equation*}

% Insert examples of blurring

Convolution follow basic properties:
\begin{description}
    \item[Commutativity] $f*g=g*f$
    \item[Associativity] $(f*g)*h = f*(g*h)$
    \item[Linearity] $(af+bg)*h = af*h + bg*h$
    \item[Shift invariance] $f_\dagger * h = (f*h)\dagger$
\end{description}
Note that is the only operator that is both linear and shift-invariant.

The convolution can be differentiated:
\begin{equation*}
    \frac{\partial}{\partial x}(f*g) = \frac{\partial f}{\partial x}*g
\end{equation*}

Gaussian filters are used to blur images.

\subsection{Computing derivatives}

\subsection{Edge detection}

\subsection{The Canny edge detector}

\subsection{Denoising, sparsity and dictionary learning}