\section{Edge detection}
\subsection{Gradient-based edge detection}
The edge detection problem aims at identifying the \emph{edges} inside an image; this requires a proper definition of \say{edge}, which often depends on the method used to compute it. Intuitively, an edge is a discontinuity of intensity in some direction; like explained previously, it could be detected by looking for place where the derivatives of the iamge have large values.

Gradient-based edge detectors run into three major issues:
\begin{enumerate}
    \item The gradient magnitudes at different scalres are different: which one should we choose?
    \item tThe gradient magnitude is large along thick trails: how do we identify the significant points?
    \item How do we link the relevant points up into curves?
\end{enumerate}

Another way to detect an extremal first derivative is to look for a null second derivative. In practice, applying a Laplacian method always require smoothing with a Gaussian kernel first. The method goes as follows: smooth the image, apply the Laplacian, and mark the zero points where there is a sufficiently large derivative, and enough contrast.

\subsection{The Canny edge detector}

\subsection{Denoising, sparsity and dictionary learning}