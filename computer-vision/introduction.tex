\section{Introduction to Computer Vision}

% - Imagerie numérique, la donner à un ordinateur, et en sort une description
% - Description sémantique : quels sont les objets de la scène
% - Description physique : associer à chaque pixel une profondeur, le matériau
% - La vision c'est difficile : ça marche parfois bien mais pas parfaitement
% - C'est difficile pour de bonnes raisons : 
%     - pour le NLP : il existe une grammaire qui est une approximation de la langue réelle, qui comporte une structure naturelle
%     - pour la vision, on ne sait pas comment marche le cerveau et la façon dont les images sont représentées
% - Autre distinction avec NLP : granularité ; Leonard de Vinci, De la perspective : la peinture est supérieure à la poésie parce qu'un texte n'est interprété qu'à un seul niveau (ou quelques) ; une image, on peut rentrer dedans à l'infini, rentrer dans les petits détails

% - Extraction d'information depuis une image/vidéo

% - La vision c'est difficile : la caméra sort une chaîne de bits
% - même nous on peut être trompés

% WHY IS VISION DIFFICULT ?
% Too much information:
% • 1000x1000x24 bits 30 times per second;
% • matching n features against n features costs n!;
% • shadows, highlights, texture..
% Too little information:
% • Physical properties (depth, orientation, reflectance..)
% of the world are not directly observable.
% What are appropriate models?
% • of images, object instances, object classes, video
% content and the interpretation process..
% What are appropriate algorithms and architectures?

% - Illusions d'optique

% COMPUTER VISION IS INTERESTING.
% • We know it is possible.
% • We know it is difficult.
% • We don’t (really) know how to do it.

% - Jan Koenderink : "fondateur" de la vision géométrique

% Why computer vision matters:
% - Safety, health, security, comfort, fun, access

% HISTOIRE
% - 1963 : thèse de L. G. Roberts : edge detection
% - 60's : interprétation d'images simples
% - 70's : segmentation

% - Face detection : gros succès pour l'autofocus, illumination...
% - Sport
% - Imagerie médicale
% - Voitures autonomes : LiDAR

% - Robotique : 

% - Generative vision : produits

% \newpage