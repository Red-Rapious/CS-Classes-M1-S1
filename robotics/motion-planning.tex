\section{Motion planning}
Motion planning is the problem of finding a sequence of valid actions that will transform the robot from its initial configuration to a desired configuration. Such actions can be constrained: we might ask the robot to avoid obstacles, to respect its kinematic limits, or to minimize energy consumption. 

We will first introduce a mathematical framework to describe the robot's configuration space, obstacles, and collision, then we will present some algorithms to solve the motion planning problem.

\subsection{Configuration space}
\subsubsection{Definitions}
The ambient space, also called the \emph{workspace}, is in most cases the Euclian manifold $\E^3$. In this ambient space, a physical system can be described as a set of compacts subsets called \emph{bodies}. The vector of the considered physical systems must be in the \emph{state space} $\S$, the set of all possible sets:
\begin{equation*}
    (K_1, \dots, K_N) \in\S \subseteq \mathcal{K}(\E^3)^N
\end{equation*}

The state space is parametrized with a transformation $q$ in the \emph{configuration space} $\Cc$:
\begin{equation*}
    q = (\varphi_1, \dots, \varphi_M)\in\Cc
\end{equation*}
where each $\varphi_i$ is a \emph{transformation} of a system, that is:
\begin{equation*}
    \varphi_i: \mathcal{K}(\E^3) \to \mathcal{K}(\E^3)
\end{equation*}
Note that $M$ corresponds to the number of systems of the robot, the other physical systems being obstacles. Therefore, given a configuration $q$, the space $S_q$ is:
\begin{equation*}
    S_q = \underbrace{\varphi_1(K_1^\circ), \dots, \varphi(K_M^\circ)}_{\text{robot}}, \underbrace{K_{M+1}, \dots, K_N}_{\text{obstacles}}
\end{equation*}
where the $K_i^\circ$ are compacts representing each body, intuitively in their own frames. We will see below that they can be represented numerically with meshes or polynomials.

In general, $\Cc$ is a manifold, its dimension being the number of degrees of freedom of the system. Note that the numerical dimension used to represent $\Cc$ is not necessarily the same as the actual dimension of the manifold: for instance, quaternions of dimension 4 can be used to represent rotations of $SO(3)$, a manifold of dimension 3.

\subsubsection{Joints}
Multiple types of joints can be used to connect the different parts of a robot. Formally, a joint is a mapping from a manifold of dimension $1\leq p\leq6$ into the solid placement $SE(3)$. The difference between the joints is the dimension of the manifold, which can also be seens as the number of degrees of freedom of the joint.
\begin{figure}
    \centering
    \includegraphics[width=.7\textwidth]{motion-planning/joints.png}
    \caption{Different types of joints.}
\end{figure}

\subsubsection{Examples}
\paragraph*{Moving solid}
Let us consider the Euclidian manifold $\E^2$ as a workspace, that is 2D objects. A solid object $K$ is described as a compact subset of $\E^2$. The configuration space is the set of transformations of this object, $SE(2)$, the special Euclidian group of dimension 2.

\paragraph*{Double pendulum}
In the case of a double pendulum, the workspace remains $\E^2$. Because of the physical constraints on the structure of the system, the pendulum is parametrized by two angles $\theta_1$ and $\theta_2$. The configuration space is therefore a torus $T^2=S^1\times S^1$, but can also be seen as $[0,2\pi)\times[0,2\pi)$.

\paragraph*{Rigid poly-articulated body}
The same general framework can be applied to more complex systems, such as humanoid robots. The workspace is $\E^3$, and the configuration $q$ of a robot is represented by the concatenation of the parameters of each joint:
\begin{equation*}
    q\in \underbrace{SE(3)}_{\text{Position}} \times \underbrace{S_1\times \dots\times S_1}_{\text{Revolute joints}} \times \underbrace{[0,1]\times\dots\times[0,1]}_{\text{Prismatic \& bounded joints}}
\end{equation*}
Recall that the forward kinematic can be used to compute the position of each joint in the global frame, given the configuration $q$.

\begin{figure}[H]
    \centering
    \begin{minipage}{.4\textwidth}
        \centering
        \includegraphics[width=.5\textwidth]{motion-planning/humanoid.png}
        \caption*{A humanoid robot}
    \end{minipage}
    \begin{minipage}{.5\textwidth}
        \centering
        \includegraphics[width=.9\textwidth]{motion-planning/tree.png}
        \caption*{The associated configuration tree}
    \end{minipage}
\end{figure}

\subsection{Obstacles and collisions}
Obstacles and collision handling are crucial in motion planning, and are responsible for the complexity of the problem. We will dive deeper into this topic in another chapter, but we introduce here the fundamental notions needed for motion planning.
\subsubsection{Representations and modelisation}
There are multiple ways to represent obstacles, which are used in different algorithms. The \emph{polynomial representation} consists in representing the obstacles as sets satisfying polynomial inequalities. The \emph{mesh representation} and \emph{primitive representation} define simple sets of points in the workspace that are respectively inside or outside the obstacles.
\begin{figure}[H]
    \centering

    \begin{minipage}{.5\textwidth}
        \centering
        \includegraphics[width=.9\textwidth]{motion-planning/polynomial.png}
        \caption*{Polynomial representation}
    \end{minipage}
    \begin{minipage}{.4\textwidth}
        \centering
        \includegraphics[width=.5\textwidth]{motion-planning/mesh.png}
        \caption*{Mesh representation}
    \end{minipage}
\end{figure}

Given a physical state $S_q$ of the form
\begin{equation*}
    S_q = \varphi_1(K_1^\circ), \dots, \varphi(K_M^\circ), K_{M+1}^\circ, \dots, K_N^\circ
\end{equation*}
we can define the space of configurations $\Cc_{\text{obs}}$ that are in collision with the obstacles:
\begin{equation*}
    \Cc_{\text{obs}} = \set{q\in\Cc}{%
        \underbrace{\exists i\neq j\leq M, \varphi_i(K_i^\circ)\cap \varphi_j(K_j^\circ)\neq\emptyset}_{\text{Auto-collision}}%
        \:\text{or}\:%
        \underbrace{\exists i\leq M, j>M, \varphi_i(K_i^\circ)\cap K_j^\circ\neq\emptyset}_{\text{Environment collision}}%
    }
\end{equation*}
We can then define the \emph{free space} $\Cc_{\text{free}}$ as the set of configurations that are not in collision with the obstacles, that is:
\begin{equation*}
    \Cc_{\text{free}} = \Cc\setminus\Cc_{\text{obs}}
\end{equation*}
Therefore, the problem of motion planning is to find a path in the free space $\Cc_{\text{free}}$ that connects the initial configuration to the final configuration.

\subsubsection{Example: translation without rotation of a rigid body}
Let us consider the workspace $\E^2$ and a rigid body $K_r$ that can be translated without rotating. The configuration space is $SE(2)$, and the obstacles are represented by a set of points in the workspace.

It is handy to compute the Minkowski sum of the robot $K_r$ and the obstacles $K_o$:
\begin{equation*}
    K_r + K_o = \set{x+y}{x\in K_r, y\in K_o}
\end{equation*}
If there is a path outside the Minkowski sum, then it is possible to find a path to move the body without rotating it.
\begin{figure}[H]
    \centering
    \includegraphics[width=.9\textwidth]{motion-planning/minkowski.png}
    \caption{Use of the Minkowski sum of the robot and the obstacles.}
\end{figure}
\begin{figure}[H]
    \centering
    \includegraphics[width=.7\textwidth]{motion-planning/minkowski-2.png}
    \caption{Another example, without a feasible path.}
\end{figure}
Thanks to the configuration space approach, motion planing can be reduced to a punctual trajectory problem. Given $q_i, q_f\in\Cc_{\text{free}}$, we want to find a continuous path $\gamma\in\Cc^0([0,1], \Cc_{\text{free}})$ such that $\gamma(0)=q_i$ and $\gamma(1)=q_f$.
\begin{figure}[H]
    \centering
    \includegraphics[width=.5\textwidth]{motion-planning/path.png}
    \caption{Example of a continuous path in the free configuration space.}
\end{figure}
Nevertheless, this still raises sub-questions. From a mathematical point of view, we can study the existence and optimality of such paths. From a numerical point of view, we need to choose a representation of the configuration space and obstacles, and to design algorithms to compute the path.

\subsection{Motion planning algorithms}
\subsubsection{Algorithm paradigms}
Three competing paradigms exist to solve the motion planning problem:
\begin{itemize}
    \item \emph{Combinatorial planning} consists in finding a global, exact planning. It is \say{moral} but computationally unrealistic.
    \item \emph{Trajectory optimization} consists in finding a local planning. It is fast but remains a local approach.
    \item \emph{Sampling-based motion planning} probabilistically explores the configuration space. It is very efficient but does not guarantee optimality. This is the most used approach in practice.
\end{itemize}

Such algorithms come with different possible level of completeness:
\begin{itemize}
    \item A \emph{complete} algorithm returns a solution if one exists, and reports a failure otherwise.
    \item A \emph{semi-complete} algorithm returns a solution if one exists, but may run forever otherwise.
    \item For a \emph{probabilistically complete} algorithm, if a solution exists, the probability that it will be found approaches 1 as the number of iterations approaches infinity.
\end{itemize}

\subsubsection{Combinatorial planning}
Combinatorial planning was developed in the 80s, and remains close to the mathematical problem studied. It is extremely efficient for low-dimensional problems, and are \emph{complete} and even sometimes \emph{resolution complete}. On the other hand, some are difficult to implement due to numerical issues, and most are intractable even with medium-sized problems.

\paragraph*{Principle}
The main idea is to build finite roadmaps; a \emph{roadmap} is a graph $\G$ such that:
\begin{itemize}
    \item each vertex is a configuration $q\in\Cc_{\text{free}}$,
    \item each edge $e(q, q')$ is a path $\gamma\in\Cc^0([0,1], \Cc_{\text{free}})$ such that $\gamma(0)=q$ and $\gamma(1)=q'$.
\end{itemize}
Furthermore, a roadmap $\G$ is a \emph{topological graph} for the space $\Cc_{\text{free}}$ if:
\begin{itemize}
    \item $\G$ is \emph{accessible}: from anywhere in $\Cc_{\text{free}}$, it is trivial to compute a path that reaches at least one point along any edge in $\G$
    \item $\G$ is \emph{homotopic-preserving}: every close path in $\Cc_{\text{free}}$ can be homotopically deformed into a path which is the concatenation of edges that form a cycle in $\G$.
\end{itemize}

In the following, we will assume that $\Cc_{\text{free}}$ is piecewise linear, that is, it is the union of a finite number of polytopes. The robot can be either a point, a polygonal, or even a disc, which has the ability to move without rotation (that is, it can be translated).
\begin{figure}[H]
    \centering
    \includegraphics[width=.35\textwidth]{motion-planning/obstacles.png}
    \caption{Example of piecewise linear obstacles.}
\end{figure}

\paragraph*{Sweep}
A first approach to build a roadmap is to use a \emph{sweep} algorithm. The idea is to use the plane sweep principle to efficiently determine where the rays terminate on the obstacles. We sort the vertices by abscisse coordinate, and we handle extensions from left to right, while maintaining a vertically sorted list of edges. This leads to a simple to implement, $O(n\log n)$-time algorithm, where $n$ is the number of vertices of the obstacles.
\begin{figure}[H]
    \centering
    \includegraphics[width=.6\textwidth]{motion-planning/sweep.png}
    \caption{Example of sweeping.}
\end{figure}

\subsubsection{Trajectory optimization (local planning)}
Trajectory optimization, also known as local planning, use optimal control methods such as DDP or iLQR to find a local solution to the motion planning problem. It has a strong terminal cost, and does not guarantee to explore all the homotopy classes. To solve this issue, we need global optimization methods, that are only viable in small dimensions. 

\subsubsection{Sampling-based motion planning (probabilistic planning)}


\newpage